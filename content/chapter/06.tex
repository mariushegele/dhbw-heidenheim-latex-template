%!TEX root = ../../main.tex

\chapter{Conclusion}

\section{Task at hand, approach and result}

Designers as well as verification engineers depend on quick and automatic builds to be able to focus on what's important - designing or verifying new logic. Building a reliable logic simulation as the basis of functional verification can be complicated and resource-intensive. This project depended on two tools to synthesize and compile a Logic design into a simulatable model. Getting a simulation to run required manual configuration and execution.

The problem was tackled by help of the build automation tool \texttt{make}. Before any implementation was possible an in-depth analysis of all steps and dependencies was necessary. To supply a continuous \texttt{Vivado} process that's requestable from within \texttt{make} a server was set up using \textit{interprocess communication} via named pipes. Availability of the \texttt{Vivado} project was provided with \textit{empty targets} and \textit{order-only prerequisites}. \textit{Intermediate} files and \textit{phony targets} were used to guarantee a clean build foundation. Functional and variable dependency mapping was achieved using meta-programming principles.

The result is a working automation of the work flow at hand that's usable without deep understanding of the actual build process. It's capable of accelerating certain builds by functioning in an \textit{incremental} manner which means nothing gets updated that doesn't have to be. But that doesn't mean that there aren't more opportunities to be employed.

\section{Possibilities of further automation and acceleration}

A simple enhancement would be to initiate the build on every new \texttt{Git} commit. This could be achieved by connecting the \texttt{make} build to a continuous integration pipeline for example using \texttt{Jenkins}. Such a pipeline triggers a build on all design changes and notifies about results or apparent errors. This allows for automatic simulation of every fresh design.

This project's \texttt{Vivado} \acs{HDL} synthesis sets up a simulation model to be compiled and executed using the \texttt{Cadence irun} utility. This offers the opportunity of using \textit{Multi-Snapshot Incremental Elaboration}  (MSIE). This function can decrease compilation times by a considerable factor. It works by defining stable or \textit{primary} partitions and variable or \textit{incremental} partitions in the \acs{DUT}. Both of these are elaborated into different snapshot that get combined into one simulation model. One benefit lies in being able to refer back to the \textit{primary} snapshot for every new design that only implements changes within the \textit{incremental} partition saving the time necessary to compile a major part of the design. Another advantage is that it enables running multiple different simulation configurations for one design. The current build is capable of only initiating one test case at a time. Using \acs{MSIE} one would be able to vary certain parts of the test-bench and run multiple different test cases (possibly at once) \cite[see][]{Cad18}.