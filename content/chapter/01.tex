%!TEX root = ../../main.tex

\chapter{Race for quantum supremacy}

Modern day computers are immensely powerful. But a rare topic of conversation are problems that are practically unsolvable for classical computers. Moore's law predicted successfully the exponential rise in power classical computers took over the last decades. It enabled us to achieve incredible feats, but there are cases in which even our best supercomputers succumb. Optimization problems like the \textit{travelling salesman} problem or the simulation of natural chemical processes classical computers can only solve inefficiently and on very small scales. But there exists a promising technological development - quantum computers.

On March 5th 2018 Google released that they succeeded in building a testable quantum computer with 72 \textit{qubits }(the equivalent of a bit in a quantum computer and the main indicator for computing power). Thereby they managed to overtake Intel's 49-qubit and IBM's 50-qubit quantum computers in the chase after \textit{quantum supremacy} - applications in which a quantum computer is able to solve problems a classical computer can't (or at least does so more efficiently) \cite[see][]{Emily18}.

Quantum mechanics poses formerly unknown problems to hardware producers. To stay relevant in competition intelligent architectural solutions are necessary. A cornerstone in constructing such architectures is functional verification of the developed design. It depends heavily on the ability to simulate the logic design in order to locate weaknesses. To guarantee quick visibility and revision these simulations best be built economically and automatically.

The goal of the project was to analyse, automate and accelerate an existing simulation build work flow for an integrated circuit that operates a quantum computer. In order to understand contextual implications of this task an inquiry into the characteristics of quantum computers and the craft of functional verification is important.